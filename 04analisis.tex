En todo el análisis se utilizarán unidades naturales...\\


Como ya se explicó en la introducción \textcolor{red}{EXPLICAR EN LA INTRODUCCIÓN} los eventos $t/\overline{t}$ + DM y $t\overline{t}$ + DM pueden contener cero, uno o dos leptones; referidos como eventos hadrónicos (AH, \textit{all hadronic}), semileptónicos (SL, \textit{single lepton}) o dileptónicos (DL, \textit{dilepton}). Este análisis se va a focalizar en los eventos $t/\overline{t}W$ + DM y $t\overline{t}$ + DM dileptónicos, centrándose en el decaimiento de dos bosones $W$ en pares electrón-neutrino o muón-neutrino. \\

A la hora de analizar los eventos es importante destacar cómo se simulan. Para ello se utiliza la herramienta de simulación de Montecarlo que permite modelar con gran precisión las colisiones protón-protón y las interaaciones que suceden en el interior del detector. En primer lugar, la generación de eventos consiste en modelar la colisión inicial y la producción de partículas a partir de las interacciones fundamentales descritas por la teoría cuántica de campos. Se utilizan herramientas como \texttt{PYTHIA} \cite{PYTHIA}, \texttt{POWHEG} \cite{POWHEG} o \texttt{MADGRAPH} \cite{MADGRAPH} . En esta etapa se incluyen fenómenos como la hadronización de los quarks y gluones, así como los decaimientos de partículas inestables según las probabilidades teóricas conocidas. Una vez los eventos son modelados, se simula cómo interactúan las partículas generadas con el detector. Para ello se utiliza \texttt{GEANT4} \cite{GEANT4}. Durante esta etapa se tienen en cuenta efectos como la pérdida de energía en los materiales, la dispersión múltiple y las respuestas de los sensores del detector, lo que permite obtener una simulación realista de los datos recolectados.\\


Durante el análisis se emplearán diversas variables leptónicas, sobre las cuales se podrán aplicar cortes para optimizar la búsqueda.


\begin{itemize}
    \item[\ding{220}] \texttt{events}: Número de eventos considerados en el análisis.

    \item[\ding{220}] \texttt{eta1} y \texttt{eta2}: Pseudorapidez del primer y segundo leptón. Se suele utilizar como coordenada espacial para indicar el ángulo con respecto al haz de partículas. 
    $$
    \eta = -\ln \left( \tan \left( \frac{\theta}{2}\right) \right)
    $$
    Donde $\theta$ es el ángulo entre el momento de la partícula $\vec{p}$ y el eje del haz.
    \item[\ding{220}] \texttt{dphill}: Diferencia en el ángulo azimutal de dos leptones.
    $$
    \Delta \phi_{ll} = |\phi_l^1-\phi_l^2|
    $$
    \item[\ding{220}] \texttt{dphillmet}: Diferencial en el ángulo azimutal entre la dirección de los leptones y la energía tranversa faltante.
    $$
    \Delta \phi_{ll,\text{MET}} = \phi_{ll}-\phi_\text{MET}
    $$
    \item[\ding{220}] \texttt{drll}: Distancia angular entre los dos leptones en el espacio $(\eta, \phi)$, definida como:
    $$
    \Delta R_{ll} = \sqrt{(\Delta \eta)^2+(\Delta \phi)^2}
    $$
    \item[\ding{220}] \texttt{mll}: Masa invariante del sistema dileptónico. Su ecuación es la siguiente:
    $$
    m_{\ell \ell} = \sqrt{(E_{\ell}^1 + E_\ell^2)^2-(\vec{p_\ell^1}+\vec{p_\ell^2})^2}
    $$
    Donde $E_\ell^i$ y $\vec{p}_\ell^i$ son la energía y el momento del leptón $i$-ésimo respectivamente.
    \item[\ding{220}] \texttt{mpmet}:  Magnitud del momento transverso faltante corregido, utilizado para mejorar la estimación de la energía faltante.
    $$
    E_T^{\text{miss}} = -\Bigg| \sum_i \vec{p_T^i} \Bigg|
    $$
    \item[\ding{220}] \texttt{puppimet}: Energía faltante calculada usando el algoritmo PUPPI, que mejora la resolución eliminando efectos del pileup.
    \item[\ding{220}] \texttt{pt1} y \texttt{pt2}: Momento del primer y segundo leptón en el eje transversal al eje del haz.
    $$
    p_T = \sqrt{p_x^2+p_y^2}
    $$
    \item[\ding{220}] \texttt{ptll}: Momento transverso del sistema dileptónico.
    $$
    p_T^{\ell \ell} = |\vec{p_T^{\ell 1}} + \vec{p_T^{\ell 2}}|
    $$
    \item[\ding{220}] \texttt{mth}: Masa transversa del sistema \textit{leading} leptón-MET, definida como:
    $$
    m_T^H = \sqrt{2p_T^{\ell 1}\cdot \text{MET}\cdot (1-\cos(\Delta \phi_{\ell 1, \text{MET}}))}
    $$
    \item[\ding{220}] \texttt{mtw2}: Masa transversal calculada considerando el segundo lepton y el MET, definida como:
    $$
    m_T^{W2} = \sqrt{2p_T^{\ell 2}\cdot \text{MET}\cdot (1-\cos(\Delta \phi_{\ell 2, \text{MET}}))}
    $$
    
    \item[\ding{220}] \texttt{mjj}: Masa invariante de dos \textit{jets} seleccionados:
    $$
    m_{j j} = \sqrt{(E_{j}^1 + E_j^2)^2-(\vec{p_j^1}+\vec{p_j^2})^2}
    $$
    Donde $E_j^i$ y $\vec{p}_j^i$ son la energía y el momento del \textit{jet} $i$-ésimo respectivamente.
\end{itemize}

Una vez definidas las variables relevantes para el análisis, es fundamental considerar los distintos procesos de fondo que pueden contribuir a los eventos observados. Estos fondos corresponden a eventos del Modelo Estándar que imitan la firma esperada de la señal y pueden dificultar su identificación.

\begin{itemize}
    \item[\ding{220}] $t\overline{t}$: Eventos de producción de un par top-antitop que decaen en dos leptones y dos neutrinos. Es decir, los quarks top decaen de la forma : $t \rightarrow W^+b$, $\overline{t} \rightarrow W^-\overline{b}$ y los bosones W decaen como $W^+ \rightarrow l^+\nu$, $W^- \rightarrow l^-\overline{\nu}$.

    \item[\ding{220}] Single top: Se refiere a la producción de un único quark top (o antitop)  en vez de un par $t\overline{t}$. Se puede dar principalmente a través de tres canales: \textit{s-channel} donde un bosón $W$ virtual decae en un par $t\overline{b}$, \textit{t-channel} donde un bosón $W$  virtual se intercambia entre un par $qb$ y $q't$ y la producción de un par $tW$ a partir de un par $gb$. \cite{SingleTop}

    \item[\ding{220}] Semi Leptonic: Este fondo se refiere a eventos $t\overline{t}$ donde uno de los quarks decae hadrónicamente y otro decae leptónicamente.

    \item[\ding{220}] $tt$V: Categoría donde se engloban los eventos con producción de un par top-antitop con un bosón $Z$, $W$ o Higgs. Se tiene en cuenta además las diferentes maneras que tiene cada bosón en decaer. Esto es, el bosón $Z$ puede decaer en 2 leptones o 2 neutrinos o puede hadronizar. El bosón $W$ puede decaer en un par leptón-neutrino o puede hadronizar.  Los eventos con Higgs pueden decaer de la siguiente manera: $t\overline{t}H \rightarrow (W^+b)(W^-\overline{b})(b\overline{b})$ y decayendo el par $t\overline{t}$ decaen por el canal dileptónico o pueden no decaer en un par $b\overline{b}$ sino en otros modos. \cite{ttV}
    
 \item[\ding{220}] VV: Producción de un par de bosones vectoriales ($W^+W^-, W^\pm Z, ZZ$). Se tienen en cuenta cuando el par de bosones $W$ decae de manera dileptónica. El par $W^\pm Z$ puede decaer en 3 leptones y un neutrino, 2 leptones y en quarks (hadronización). Por último, el par $ZZ$  puede decaer en 2 leptones y 2 neutrinos, se puede hadronizar y decaer en quarks y 2 leptones o en 4 leptones.

 \item[\ding{220}] Drell-Yan (DY): El proceso de Drell-Yan ocurre cuando se aniquila un quark-antiquark en colisiones hadrónicas, creando un bosón $Z$ o un fotón virtual que decaen en un par de leptones con carga opuesta. \cite{wiki:Drell–Yan_process}

 \item[\ding{220}] \textit{Other}: En esta categoría se incluyen eventos que pueden ser de interés pero que no se corresponden a ningún otro grupo de eventos. Por ejemplo, pares $Z\gamma$ que decaen en 2 leptones y un fotón o eventos con Higgs.
\end{itemize}


\section{Adaptación del análisis}

El \textit{framework} del trabajo estaba inicialmente orientado a la búsqueda de señales de WW + DM en los datos, con selección y cortes optimizados para ese canal. \cite{CMSWW} Sin embargo, el enfoque se ha modificado para centrarse en la búsqueda de firmas de producción de pares de quarks top-antitop. Esto ha implicado ajustes en los criterios de selección de eventos, en la definición de regiones de señal y fondo, y en la estrategia general del análisis para maximizar la sensibilidad a esta nueva hipótesis. Mientras que en la búsqueda de  WW+DM se priorizaban eventos con estados finales leptónicos y MET significativo, el análisis de $t\overline{t}$ requiere una identificación precisa de b-jets y una reconstrucción más detallada de los decaimientos hadrónicos, semileptónicos o dileptónicos de los quarks top.\\

Durante todo el análisis se va a utilizar un modelo teórico de materia oscura simplificada. Se asumen que las partículas de materia oscura $\chi$ son fermiones y los mediadores $\phi$ son partículas con \textit{spin} 0. Las constantes de acoplamiento tanto entre el mediador y las partículas del modelo estándar $g_q$ como entre las partículas mediadoras con las partículas fermiónicas de materia oscura $g_\chi$ se asume por simplicidad $g_q = g_\chi = 1$. Los dos parámetros que quedan libres son las masas de las partículas $\chi$ y $\phi$. Se van a definir 17 puntos de masa distintos: variando la masa de la partícula mediadora entre 50 a 500 GeV en pasos de 50 GeV dejando fija la masa de la partícula de materia oscura a $m_\chi = 1$ GeV y fijando la masa del mediador $m_\phi = 100$ GeV y variando la masa del fermión $m_\chi = \{ 20, 30 ,40 ,45, 49 ,51, 55 \}$ GeV. De ahora en adelante, los puntos de masa se representarán con dos números en un paréntesis separados con un símbolo de coma, de manera que ($m_\chi$, $m_\phi$).\\

En la figura \ref{fig:comparacion_analisisWW_tt} se puede observar un \textit{stack plot} donde se ve la diferencia entre nuestro análisis y en el que nos hemos inspirado. Los \textit{stack plots} son un tipo de gráficas que se suele utilizar en la rama de la física de partículas donde se muestra las diferentes contribuciones a una distribución de manera que los fondos aparecen apilados unos sobre otros. En el eje x aparece la variable que se estudia (en este caso, la masa invariante) dividida en pequeños intervalos llamados \textit{bins}. Cada proceso físico ($t\overline{t}$, DY, $tt$V...) se representa con un histograma de distinto color, de manera que la altura total de cada \textit{bin} representa el número total de eventos esperados. Las señales se muestran con líneas sin apilar. Aparte de los histogramas apilados aparecen puntos de datos experimentales para comparar la diferencia entre los eventos simulados y los eventos reales, también llamado \textit{mismodelling}. En este caso se representa la masa invariante de los leptones en ambos casos. Se puede observar el cambio en los eventos a analizar y que se pinta en la gráfica dos regiones de señal: en azul cían para la producción de partículas de materia oscura con masa (1, 50) y en naranja para partículas con masa (1, 500).
 

\begin{figure}[h!]
     \centering
     \begin{subfigure}[b]{0.43\textwidth}
         \centering
         \includegraphics[width=\textwidth]{images/log_cratio_dhww2l2v_13TeV_sr_1bj_mll_ANALISISWW.png}
         \caption{Signatura $WW$ + DM}
         \label{fig: ANALISISWW}
     \end{subfigure}
     \begin{subfigure}[b]{0.43\textwidth}
         \centering
         \includegraphics[width=\textwidth]{images/log_cratio_dhww2l2v_13TeV_sr_emu_1bj_mll.png}
         \caption{Signatura $t\overline{t}$ + DM}
         \label{fig: ANALISISttbar}
     \end{subfigure}
     \hfill
     \caption{Comparación entre las categorías del análisis con signatura $WW$ + DM  y las categorías del análisis con signatura $t\overline{t}$ + DM.}
     \label{fig:comparacion_analisisWW_tt}

\end{figure}



\section{Selección de sucesos básica}

Uno de los objetivos del trabajo es intentar maximizar la señal que queremos observar frente al fondo total. Para ello se aplican cortes y selecciones en las distintas variables y se impone tener en cuenta ciertos eventos. Todo el análisis va a estar inspirado en el \textcolor{cyan}{Analysis Note}. A partir de ahora se considerarán los eventos $t/\overline{t}$ + DM y $t\overline{t
}$ que contengan dos leptones (canal DL).\\

El análisis se lleva a cabo en un principio en tres canales distintos: canal electrón-electrón $ee$, canal electrón-muón $e\mu$ y canal muón-muón $\mu \mu$. Para esta selección se impone que los dos leptones que se generan tengan carga opuesta y se excluyen eventos unileptónicos. También se fuerza que el $p_T$ del leptón \textit{leading} (el que tiene el mayor $p_T$) sea mayor que 25 GeV y el leptón \textit{trailing} (el que tiene el segundo mayor $p_T$) sea mayor que 20 GeV. Esto se hace para que no se tenga en cuenta las regiones con $p_T$ pequeñas ya que así se reduce contribuciones al fondo y se excluyen zonas de baja eficiencia del detector. Además, si hay 3 leptones se excluyen los eventos cuyo $p_T$ del tercer leptón sea mayor de 10 GeV. Así se evitan eventos donde haya 3 o más leptones significativos. Se impone además un corte en $m_{ll}$, que tiene que ser mayor que 20 GeV para evitar resonancias a bajas energías. Para que se reduzca aún más el fondo DY, en los canales $ee$ y $e\mu$ se aplica un veto en $m_{ll}$ alrededor de la masa del bosón Z. Finalmente se aplica un corte a mpmet y puppimet para...........\\

Dentro de estas preselecciones se crean diferentes categorías atendiendo al número de jets que se crean a partir de la hadronización de un quark bottom (b-jets). La condición para que un jet se considere b-jet es que su momento transverso sea mayor de 30 GeV, debe tener una pseudorapidez de $|\eta| < 2.5$ y un valor del discriminante de b-jets (una manera de indicar cómo de probable es de confundir un jet por un b-jet) de 0.1241, correspondiente a un \textit{Medium Working Point}. \cite{LooseWorkingPoint} Para cada par leptónico ($ee$, $e\mu$,  $\mu\mu$) y dependiendo cuantos b-jets se han reconocido se han definido 5 categorías: la categoría \textbf{in} no impone ninguna condición sobre el número de b-jets, \textbf{0bj} cuando no se ha reconocido ningún b-jet, \textbf{eq1bj} cuando se ha reconocido exactamente 1, \textbf{1bj} cuando se ha reconocido 1 o más y \textbf{2bj} cuando se ha reconocido 2 o más. Como se puede esperar, la categoría \textbf{in} es la más inclusiva. 

\begin{table}[h!]
\centering
\begin{tabular}{|ccccc|}
\hline
\multicolumn{5}{|c|}{\textbf{Selección básica}}                                                                               \\ \hline

\multicolumn{1}{|c|}{\textbf{in}} & \multicolumn{1}{c|}{\textbf{0bj}}          & \multicolumn{1}{c|}{\textbf{1bj}}            & \multicolumn{1}{c|}{\textbf{eq1bj}}        & \multicolumn{1}{c|}{\textbf{2bj}}          \\ \hline

\multicolumn{1}{|c|}{$n_{bJ} \ge 0$} & \multicolumn{1}{c|}{$n_{bJ} = 0$} & \multicolumn{1}{c|}{$n_{bJ} \ge 1$} & \multicolumn{1}{c|}{$n_{bJ} = 1$} & \multicolumn{1}{c|}{$n_{bJ} \ge 2$}    \\ \hline
\multicolumn{5}{|c|}{$q_{l1} \cdot q_{l2} < 0$}                                                                                 \\ \hline
\multicolumn{5}{|c|}{$n_l \ge 2$}                                                                                             \\ \hline
\multicolumn{5}{|c|}{$p_T^{l1} > 25 $ GeV}                                                                                       \\ \hline
\multicolumn{5}{|c|}{$p_T^{l2} > 20 $ GeV}                                                                                       \\ \hline
\multicolumn{5}{|c|}{Si $n_l > 2$, $p_T^{l3} < 10$ GeV}                                                                          \\ \hline
\multicolumn{5}{|c|}{$m_{ll}> 20$ GeV, si el sabor del leptón es el mismo: $|m_{ll}-m_Z| > 15 $ GeV}                           \\ \hline
\multicolumn{5}{|c|}{mpmet $> 20$ GeV}                                                                                        \\ \hline
\multicolumn{5}{|c|}{PUPPImet $>$ 20 GeV}                                                                                     \\ \hline

\end{tabular}
\caption{\textcolor{cyan}{Cambiar el formato}}
\label{tab:preseleccion_basica}
\end{table} 



En la figura \ref{fig:comparacion_preseleccion_basica} se muestran los \textit{stack plots} de la $m_{\ell \ell}$ y $p_T^{\ell \ell}$ para alguna de las regiones de esta selección. Se puede observar en las gráficas correspondientes a $m_{\ell \ell}$ los cortes en la región del bosón Z. También se puede destacar cómo en las categorías $ee$ y $\mu \mu$ el fondo de Drell-Yan es más importante que en el caso de $e \mu$, debido a producciones de pares leptónicos del mismo sabor por fotones virtuales o por la resonancia del bosón Z.\\


\begin{figure}[h!]
     \centering
     \begin{subfigure}[b]{0.32\textwidth}
         \centering
         \includegraphics[width=\textwidth]{images/log_cratio_dhww2l2v_13TeV_sr_ee_1bj_mll.png}
         \caption{Categoría $ee$\_in}
         \label{fig: ANALISISWW}
     \end{subfigure}
     \begin{subfigure}[b]{0.32\textwidth}
         \centering
         \includegraphics[width=\textwidth]{images/log_cratio_dhww2l2v_13TeV_sr_mumu_1bj_mll.png}
         \caption{Categoría $\mu \mu$\_in}
         \label{fig: ANALISISttbar}
     \end{subfigure}
     \begin{subfigure}[b]{0.32\textwidth}
         \centering
         \includegraphics[width = \textwidth]{images/log_cratio_dhww2l2v_13TeV_sr_emu_1bj_mll.png}
         \caption{Categoría $e \mu$\_in}
         \label{}
     \end{subfigure}
     \hfill
    \begin{subfigure}[b]{0.32\textwidth}
        \centering
         \includegraphics[width = \textwidth]{images/log_cratio_dhww2l2v_13TeV_sr_ee_ptll.png}
         \caption{Categoría $e \mu$\_in}
         \label{}
     \end{subfigure}
         \begin{subfigure}[b]{0.32\textwidth}
        \centering
         \includegraphics[width = \textwidth]{images/log_cratio_dhww2l2v_13TeV_sr_mumu_ptll.png}
         \caption{Categoría $e \mu$\_in}
         \label{}
     \end{subfigure}
         \begin{subfigure}[b]{0.32\textwidth}
        \centering
         \includegraphics[width = \textwidth]{images/log_cratio_dhww2l2v_13TeV_sr_emu_ptll.png}
         \caption{Categoría $e \mu$\_in}
         \label{}
     \end{subfigure}
     
     \caption{Histogramas obtenidos para la selección básica. En la fila de arriba se representan las gráficas para la variable $m_{\ell \ell}$ y en la fila de abajo para la variable $p_T^{\ell \ell}$.}     \label{fig:comparacion_preseleccion_basica}

\end{figure}

Durante el análisis se ha visto que para las categorías $ee$ y $\mu\mu$ el \textit{mismodelling} para energía transversa perdida era bastante significativo, esto puede ser debido a inconsistencias en la determinación de los fondos con un leptón genuino y uno instrumental. Con el fin de sacar mejores conclusiones en el estudio, a partir de ahora solo se tendrán en cuenta los eventos de la categoría $e \mu$.\\


\section{Selección del análisis}

Para la selección del análisis se va a utilizar únicamente eventos que ocurran en el canal $e\mu$ cuyas cargas sean opuestas. Se van a definir tres regiones atendiendo al número de b-jets: $t/\overline{t}$ + DM para eventos con un único b-jet, $t\overline{t}$ + DM para eventos con 2 o más b-jets y $tt\overline{t}$ + DM para el caso más inclusivo donde hay 1 o más b-jets.  A continuación se imponen ciertos cortes en los momentos transversos para los leptones \textit{leading} y \textit{trailing} para que sean mayor que 25 GeV y 20 GeV respectivamente. Además, se pide que el número de leptones sea exactamente igual a 2. Por último se toman los eventos cuyo $m_{ll}$ sea mayor que 20 GeV y los eventos donde al menos haya un jet.\newpage

\begin{table}[h!]
\centering
\begin{tabular}{|ccc|}
\hline
\multicolumn{3}{|c|}{\textbf{Selección del análisis}}                                                  \\ \hline
\multicolumn{1}{|c|}{$t$\textbf{DM}} & \multicolumn{1}{c|}{$tt$\textbf{DM}} & \textbf{$tt\overline{t}$DM} \\ \hline
\multicolumn{1}{|c|}{$n_{bJ} = 1$}       & \multicolumn{1}{c|}{$n_{bJ} \ge 2$}    & $n_{bJ} \ge 1$     \\ \hline
\multicolumn{3}{|c|}{Canal $e\mu$, $q_{l1} \cdot q_{l2} < 0$}                                            \\ \hline
\multicolumn{3}{|c|}{$n_l = 2$}                                                                        \\ \hline
\multicolumn{3}{|c|}{$n_J \ge 1$} 
                        \\ \hline
\multicolumn{3}{|c|}{$p_T^{l1} > 25 $ GeV}                                                                \\ \hline
\multicolumn{3}{|c|}{$p_T^{l2} > 20 $ GeV}                                                                \\ \hline
\multicolumn{3}{|c|}{Si $n_l > 2$, $p_T^3 < 10$ GeV}                                                   \\ \hline
\multicolumn{3}{|c|}{$m_{ll}> 20$ GeV}                                                                 \\ \hline

\end{tabular}
\caption{Selección básica aplicada a los procesos del análisis.}
\label{tab:preseleccion_anlisis}
\end{table}

En la figura \ref{fig:Stack_PreseleccionAN} se muestran los histogramas para algunas variables de esta selección. En este caso no se va a hacer hincapié en los datos experimentales y por tanto se representa unicamente la simulación.\newpage
\begin{figure}[h!]
     \centering
     \begin{subfigure}[b]{0.43\textwidth}
         \centering
         \includegraphics[width=\textwidth]{images/log_cratio_dhww2l2v_13TeV_sr_emu_tttDM_ptll.png}
         \caption{Variable \texttt{ptll}}
         \label{fig:stack_tttDM_ptll}
     \end{subfigure}
     \begin{subfigure}[b]{0.43\textwidth}
         \centering
         \includegraphics[width=\textwidth]{images/log_cratio_dhww2l2v_13TeV_sr_emu_tttDM_mll.png}
         \caption{Variable \texttt{mll}}
         \label{fig:stack_tttDM_mll}
     \end{subfigure}
     \hfill
          \begin{subfigure}[b]{0.43\textwidth}
         \centering
         \includegraphics[width=\textwidth]{images/log_cratio_dhww2l2v_13TeV_sr_emu_tttDM_mpmet.png}
         \caption{Variable \texttt{mpmet}}
         \label{fig:stack_tttDM_mpmet}
     \end{subfigure}
          \begin{subfigure}[b]{0.43\textwidth}
         \centering
         \includegraphics[width=\textwidth]{images/log_cratio_dhww2l2v_13TeV_sr_emu_tttDM_mtw2.png}
         \caption{Variable \texttt{mtw2}}
         \label{fig:stack_tttDM_mtw2}
     \end{subfigure}
     \caption{\textcolor{red}{VARIABLE MPMET TIENE 'DOS SEÑALES'} Histogramas para algunas variables de la categoría tttDM del análisis. }
     \label{fig:Stack_PreseleccionAN}
\end{figure}

Se puede observar que en las variables \texttt{mpmet} y \texttt{mtw2} el comportamiento de fondo y señal difieren mucho. Esto nos sugiere sobre qué variable centrarnos en buscar cortes óptimos.

\section{Optimización de la señal y automatización}

Con el fin de optimizar los cortes para maximizar la sensibilidad a la señal que se busca se estudiará la figura de mérito (FOM, por sus siglas en inglés \textit{Figure of Merit}) de las distintas distribuciones. Este procedimiento es muy recurrente en el contexto de la física de partículas, ya que minimiza la contribución del fondo en nuestro análisis. La figura de mérito tiene la siguiente expresión:

\begin{equation}\label{eq: fom}
    \text{FOM} = \frac{S}{\sqrt{S+B}}
\end{equation}

Donde $S$ son los sucesos de señal y $B$ son los sucesos de fondo. La idea de este apartado es calcular la figura de mérito en dos variables: \texttt{mpmet} y \texttt{mtw2}. Se iterará para cada \textit{bin} y se encontrará el valor de la variable que maximiza la figura de mérito. Estos valores serán los cortes óptimos para mpmet y mtw2.\\


Se ha creado dos códigos para el cálculo de la figura de mérito, que se exponen en el Apéndice \textcolor{cyan}{NÚMERO}, llamados \texttt{1D$\_$opti$\_$fom.py} y \texttt{2D$\_$opti$\_$fom.py}. En nuestro análisis nos va a interesar optimizar 2 variables: \texttt{mpmet} y \texttt{mtw2}. La diferencia entre ambos códigos es que el primero se utiliza para optimizar únicamente una de las dos variables, mientras que el segundo código se utiliza para optimizar las dos a la vez con el objetivo de ahorrar un poco de tiempo. Se ha tomado como eventos de señal las correspondientes a la creación de partículas de materia oscura para diferentes puntos de masa. \\

El algoritmo para el cálculo de la figura de mérito se presenta a continuación. Inicialmente se definen los argumentos que se pasarán por terminal a la hora de correr el código. El código abre el archivo de \texttt{ROOT} que contiene los histogramas y recorre todos los directorios correspondientes a las distintas categorías. Una vez localizada la categoría que se quiere estudiar, se accede al directorio correspondiente y se recorren todos los \textit{bins} que componen los histogramas. Si el nombre del histograma es el mismo que el de la región de señal, se almacena en la variable \texttt{SIGNAL} el valor de la integral del histograma. La integral puede calcularse de dos maneras: acumulando desde el $bin$ inicial hasta el $bin$ actual (integral desde la izquierda) o desde el $bin$ actual hasta el $bin$ final (integral desde la derecha). Si el nombre del histograma no coincide con el de la señal, el valor de la integral se le suma a la variable \texttt{TOTAL}. Tras esto, se calcula el valor de la figura de mérito según la ecuación \ref{eq: fom} y se comprueba si es su valor máximo hasta ese momento. En caso afirmativo, se almacena el valor máximo de la FOM junto con el corte óptimo, la cantidad de eventos de señal y de fondo. El código además tiene un par de funciones para crear las gráficas de la figura de mérito y los \textit{shapes plots}.\\

Inicialmente se ha calculado la figura de mérito para las dos variables de manera independiente, obteniendo los siguientes resultados:\\

\begin{figure}[h!]
     \centering
     \begin{subfigure}[b]{0.43\textwidth}
         \centering
         \includegraphics[width=\textwidth]{images/emu__tttDM_FOM_mpmet_mchi1_mphi100.png}
         \caption{Variable mpmet}
         \label{fig: ANALISISWW}
     \end{subfigure}
     \begin{subfigure}[b]{0.43\textwidth}
         \centering
         \includegraphics[width=\textwidth]{images/emu__tttDM_FOM_mtw2_mchi1_mphi100.png}
         \caption{Variable mtw2}
         \label{fig: ANALISISttbar}
     \end{subfigure}
     \hfill
     \caption{Gráficas de la figura de mérito para las dos variables a analizar en la categoría \_tttDM y punto de masa (1, 100).}
     \label{fig:FOM_sincortes_mpmet_mtw2}
\end{figure}

En la tabla \ref{tab:fom_binxbin_tttDM_1_100} se refleja el análisis de la figura de mérito realizado sobre la muestra punto de masa es (1, 100) para la variable \texttt{mpmet} y categoría $\_$tttDM. Se ve que el valor de \texttt{mpmet} que mejor caracteriza la señal es \texttt{mpmet} = 110 GeV.  



\begin{table}[h!]
\centering
\begin{tabular}{ |c|c|c|c|c| }
 \hline
Corte del \textit{bin} [GeV]  & S (Señal) &  Fondo(S+B)  & FoM &  Corte óptimo [GeV] \\    \hline
0.00 & 2291.99 & 351298.18 & 3.87 & 0.00 \\
 \hline
10.00 & 2050.47 & 317032.83 & 3.64 & 0.00 \\
 \hline
20.00 & 1864.42 & 283576.66 & 3.50 & 0.00 \\
 \hline
30.00 & 1705.47 & 240643.33 & 3.48 & 0.00 \\
 \hline
40.00 & 1557.08 & 197946.38 & 3.50 & 0.00 \\
 \hline
50.00 & 1414.15 & 156504.48 & 3.57 & 0.00 \\
 \hline
60.00 & 1275.58 & 118180.96 & 3.71 & 0.00 \\
 \hline
70.00 & 1140.85 & 85151.33 & 3.91 & 70.00 \\
 \hline
80.00 & 1015.29 & 59087.08 & 4.18 & 80.00 \\
 \hline
90.00  & 892.22 & 40406.31 & 4.44 & 90.00 \\
 \hline
100.00 & 778.15 & 28039.25 & 4.65 & 100.00 \\
 \hline
110.00 & 675.75 & 20182.67 & 4.76 & 110.00 \\
 \hline
120.00 & 581.77 & 15238.73 & 4.71 & 110.00 \\
 \hline
130.00 & 499.01 & 11999.99 & 4.56 & 110.00 \\
 \hline
140.00 & 426.76 & 9750.00 & 4.32 & 110.00 \\
 \hline
150.00 & 364.28 & 8102.34 & 4.05 & 110.00 \\
 \hline
160.00 & 307.57 & 6823.53 & 3.72 & 110.00 \\
 \hline
170.00 & 260.32 & 5790.59 & 3.42 & 110.00 \\
 \hline
180.00 & 220.83 & 4942.90 & 3.14 & 110.00 \\
 \hline
190.00 & 187.40 & 4234.90 & 2.88 & 110.00 \\
 \hline
200.00 & 158.79 & 3632.04 & 2.63 & 110.00 \\
 \hline
210.00 & 133.18 & 3121.74 & 2.38  & 110.00 \\
 \hline
220.00 & 112.93 & 2677.86 & 2.18 & 110.00 \\
 \hline
230.00 & 94.43 & 2303.46 & 1.97 & 110.00 \\
 \hline
240.00 & 80.83 & 1993.71 & 1.81 & 110.00 \\
 \hline
250.00 & 68.11 & 1713.72 & 1.65 & 110.00 \\
 \hline
260.00 & 57.90 & 1483.86 & 1.50 & 110.00 \\
 \hline
270.00 & 48.70 & 1281.27 & 1.36 & 110.00 \\
 \hline
280.00 & 42.24 & 1107.54 & 1.27 & 110.00 \\
 \hline
290.00 & 36.65 & 960.57 & 1.18 & 110.00 \\
 \hline

\end{tabular}
\caption{Valor del corte del \textit{bin}, el número de eventos de señal $S$, el número total de eventos del fondo $S+B$, el valor de la figura de mérito según la ecuación \ref{eq: fom} y el valor del corte óptimo hasta el \textit{bin} correspondiente.}
\label{tab:fom_binxbin_tttDM_1_100}
\end{table}


Como se puede ver, la figura de mérito tiene su máximo alrededor de 110 GeV para la variable mpmet y \textcolor{red}{120} GeV para la variable mtw2. Se utilizarán estos valores como cortes óptimos ya que maximizan la sensibilidad de la señal.\\


Como se tienen 17 regiones de señal, sería conveniente realizar esto para todas ellas. Mediante el uso de un nuevo archivo \texttt{opti\_all.sh} se ha iterado la búsqueda de los cortes óptimos en las 2 variables \texttt{mpmet} y \texttt{mtw2} para todos los puntos de masa. A continuación se muestra una tabla con los cortes óptimos para todos los puntos de masa, junto a los eventos de señal y fondo y el valor de la figura de mérito.
\newpage

\begin{table}[h!]
\centering
\begin{tabular}{ |c| |c| |c| |c| |c| }
 \hline
Punto de masa [GeV] & Corte óptimo [GeV] & Señal(S) & Fondo(S+B) & FoM \\
 \hline
(1, 50) & 110.00 & 570.79 & 20182.67 & 4.02 \\
 \hline
(1, 100) & 110.00 & 675.75 & 20182.67 & 4.76 \\
 \hline
(1, 150) & 120.00 & 640.76 & 15238.73 & 5.19 \\
 \hline
(1, 200) & 120.00 & 639.39 & 15238.73 & 5.18 \\
 \hline
(1, 250) & 120.00 & 695.62 & 15238.73 & 5.64 \\
 \hline
(1, 300) & 130.00 & 649.03 & 11999.99 & 5.92 \\
 \hline
(1, 350) & 130.00 & 660.75 & 11999.99 & 6.03 \\
 \hline
(1, 400) & 130.00 & 676.99 & 11999.99 & 6.18 \\
 \hline
(1, 450) & 130.00 & 702.39 & 11999.99 & 6.41 \\
 \hline
(1, 500) & 130.00 & 708.54 & 11999.99 & 6.47 \\
 \hline
(20, 100) & 110.00 & 677.82 & 20182.67 & 4.77 \\
 \hline
(30, 100) & 110.00 & 675.32 & 20182.67 & 4.75 \\
 \hline
(40, 100) & 110.00 & 678.00 & 20182.67 & 4.77 \\
 \hline
(45, 100) & 110.00 & 678.35 & 20182.67 & 4.77 \\
 \hline
(49, 100) & 120.00 & 575.90 & 15238.73 & 4.67 \\
 \hline
(51, 100) & 120.00 & 609.80 & 15238.73 & 4.94 \\
 \hline
(55, 100) & 120.00 & 650.36 & 15238.73 & 5.27 \\
 \hline

\end{tabular}
\caption{Cortes óptimos para cada punto de masa con su correspondientes eventos de señal, fondo y figura de mérito para la variable \texttt{mpmet} en la categoría tttDM.}
\label{tab:auto}
\end{table}

Otra manera de analizar los datos es utilizar un \textit{shape plot} en vez de un \textit{stack plot}. Los \textit{shape plots} son gráficos donde se muestra la forma normalizada de una distribución en vez de mostrar el número absoluto de eventos. Esto se realiza con el fin de evaluar la discriminación entre fondo y señal y comparar la forma de la distribución en diferentes regiones. Al igual que en los \textit{stack plots}, en el eje x se muestra la variable a analizar. Sin embargo, en el eje y aparecen la frecuencia normalizada de los eventos de señal y de fondo. Esto se realiza, al igual que con la figura de mérito, para los eventos de señal y fondo de las variables \texttt{mpmet} y \texttt{mtw2} los distintos puntos de masa.\\

\begin{figure}[h!]
     \centering
     \begin{subfigure}[b]{0.43\textwidth}
         \centering
        \includegraphics[width=\textwidth]{images/Shapes_mpmet__tttDM_mchi1_mphi100.png}
         \caption{Variable \texttt{mpmet}}
         \label{fig: Shapes_signal_bckg_mpmet}
     \end{subfigure}
     \begin{subfigure}[b]{0.43\textwidth}
         \centering
         \includegraphics[width=\textwidth]{images/Shapes_mtw2__tttDM_mchi1_mphi100.png}
         \caption{Variable \texttt{mtw2}}
         \label{fig: Shapes_signal_bckg_mtw2}
     \end{subfigure}
     \hfill
     \caption{\textit{Shape plots} para las dos variables \texttt{mpmet} y \texttt{mtw2} para el punto de masa (1, 100) en la categoría \_tttDM. En azul se refleja los eventos (normalizados) del fondo mientras que en rojo se muestra los eventos de señal.}
     \label{fig:Shapes_signal_bckg}
\end{figure}

Se puede ver claramente varias regiones diferenciadas. Si nos fijamos, por ejemplo, en la figura \ref{fig: Shapes_signal_bckg_mtw2} se ve como en la región de bajos valores hay una gran contaminación del fondo, por lo que no es ideal para la selección de eventos de señal. Alrededor de 120 GeV el número de eventos de señal y de fondo se comienza a igualar, y en valores grande la señal comienza a dominar. Esto sugiere dónde realizar el corte para mejorar la selección de señal.\\


Para evaluar el rendimiento de dos variables, se calculó la integral de la figura de mérito  en dos regiones distintas: la región donde predomina el fondo (FOM$_\text{f}$) y la región donde predomina la señal (FOM$_\text{s}$). Las regiones se dividieron en los rangos de 80 GeV y 110 GeV para \texttt{mpmet} y \texttt{mtw2} respectivamente.

Los resultados obtenidos fueron los siguientes:

\begin{itemize}
    \item Para la variable \texttt{mpmet}:
        \begin{itemize}
            \item FOM$_\text{f}$ = 33.37
            \item FOM$_\text{s}$ = 62.63
        \end{itemize}
    \item Para la variable \texttt{mtw2}:
        \begin{itemize}
            \item FOM$_\text{f}$ = 43.42
            \item FOM$_\text{s}$ = 45.17
        \end{itemize}
\end{itemize}

La figura de mérito total se calculó mediante la suma cuadrática de los valores obtenidos en ambas regiones:

\begin{itemize}
    \item FOM$_\text{mpmet}$ = 70.96
    \item FOM$_\text{mtw2}$ = 62.66
\end{itemize}

Para analizar y comparar distintas regiones donde la señal y el fondo tienen diferentes comportamientos, se presentan los siguientes \textit{shape plots}. En particular, se pone un enfoque especial en el fondo de $t\bar{t}$, ya que constituye la principal fuente de contribución en nuestro análisis. Al igual que antes, estos gráficos permiten visualizar cómo varían las distribuciones normalizadas de las diferentes señales en relación con el fondo, facilitando la identificación de posibles regiones donde la separación entre ambas sea más efectiva.\\
\newpage

\begin{figure}[h!]
     \centering
     \begin{subfigure}[b]{0.43\textwidth}
         \centering
         \includegraphics[width=\textwidth]{images/Shapes_mpmet__tttDM_4histos.png}
         \caption{Variable \texttt{mpmet}}
         \label{fig: Shapes_4histos_mpmet_tttDM}
     \end{subfigure}
     \begin{subfigure}[b]{0.43\textwidth}
         \centering
    \includegraphics[width=\textwidth]{images/Shapes_mtw2__tttDM_4histos.png}
         \caption{Variable \texttt{mtw2}}
         \label{fig: Shapes_4histos_mtw2_tttDM}
     \end{subfigure}
     \hfill
     \caption{\textit{Shape plots} para las dos variables \texttt{mpmet} y \texttt{mtw2} para los punto de masa (1, 50) en rojo, (1, 100) en azul, (1,250) en verde y el fondo $t\bar{t}$ en rosa para la categoría \_tttDM.}
     \label{fig:Shapes_4histos}
\end{figure}

El fondo en la figura \ref{fig: Shapes_signal_bckg_mpmet} y \ref{fig: Shapes_4histos_mpmet_tttDM} presenta un pico para valores bajos. Esto es esperable para desintegración por el canal semileptónico del par top-antitop ya que aparece un neutrino que aporta a la energía transversa perdida. La energía de este neutrino no suele ser muy alta, por lo que es esperable que la MET por el canal $t\bar{t}$ decaiga rápidamente. También es esperable que para las señales la distribución de MET sea mayor, ya que la producción de partículas de materia oscura aporta a la energía perdida al no interactuar con el detector.\\

Igualmente, se observa un pico del fondo en las figuras \ref{fig: Shapes_signal_bckg_mtw2} y \ref{fig: Shapes_4histos_mtw2_tttDM} alrededor de 80 GeV. Esto concuerda con la masa del bosón $W$ ya que la variable \texttt{mtw2} trata de encontrar el valor mínimo posible de la masa transversa para el sistema $W$ suponiendo que la MET proviene de dos partículas invisibles. El número de eventos cuando los valores de la variable se hacen más grandes disminuyen rápidamente debido a que el canal $t\bar{t}$ no tiene la suficiente energía como para producir valores grandes de \texttt{mtw2}. Sin embargo, la señal de materia oscura y sobretodo para masas grandes sí que sería capaz de producir magnitudes mayores, por lo que se corrobora la utilidad de estudiar esta variable.\\

Se ha realizado también las optimizaciones de los cortes de manera secuencial. La idea es la siguiente: Se aplica la preselección del análisis, se calcula el corte óptimo para una de las variables mediante el código de la figura de mérito y ese corte se aplica. A continuación se mandan 866 tareas de procesamiento de datos, llamados \textit{jobs}, a la red de computación del CERN donde se ejecutan. Una vez concluyen se devuelven los datos procesados, se vuelve a optimizar la segunda variable, se aplica este corte y se vuelven mandar los \textit{jobs} y se reejecutan. Así, se puede ver si aplicar el corte en una variable y después en la otra afecta algo que si se hiciese al revés. \\



Para la automatización de los cortes se han creado dos archivos: \texttt{applycut.py} y \texttt{auto$\_$opticut.py}. Ambos archivos serán expuestos en el apéndice \textcolor{cyan}{NÜMERO APËNDICE}. El primer fichero tiene como objetivo modificar el archivo donde se definen los cortes de las variables y las diferentes categorías para introducir el corte óptimo obtenido mediante la figura de mérito en la variable estudiada de manera automática. El archivo toma como argumentos el nombre del archivo que se quiere modificar (en nuestro \textit{framework}, \texttt{cuts.py}), el nombre de la variable que se quiere modificar \texttt{var$\_$name} y el valor del corte en dicha variable \texttt{cut$\_$value}. El código accede a la variable \texttt{$\_$tmp} del archivo \texttt{cuts.py}, que es donde se aplican los cortes. Una vez realizado esto busca si la variable \texttt{var} ya se encuentra en \texttt{$\_$tmp}. Si se encuentra el corte en esa variable, lo reemplaza por el valor de \texttt{cut$\_$value}. Si no se localiza, se añade una línea donde se aplica el corte. Este archivo devuelve el fichero con los nuevos cortes aplicados.\\

Por otro lado, el archivo \texttt{auto$\_$opticut.py} busca el valor del corte óptimo en la variable que se quiere estudiar y aplica en el archivo \texttt{cuts.py}. Para ello toma por terminal 4 argumentos: los valores de $m_\chi$ y $m_\phi$ y la variable y la categoría que se quieren estudiar (\texttt{MCHI, MPHI, VAR} y \texttt{CAT} respectivamente). El archivo llama al fichero \texttt{1D\_opti\_FOM.py} donde se calcula el valor óptimo de la figura de mérito para una variable, y este valor es escrito en un archivo de texto. Posteriormente, se accede a este nuevo archivo de texto para guardar el valor del corte óptimo y se utiliza \texttt{applycuts.py} para cambiar \texttt{cuts.py}. Finalmente, el código manda los \textit{jobs} al \textit{grid} del CERN para ser ejecutados. \\

Inicialmente, al aplicar un primer corte en la variable \texttt{mpmet}, se determina que el valor óptimo para maximizar la separación entre señal y fondo es de 110 GeV. Una vez establecido este umbral, se procede a evaluar el siguiente corte en la variable \texttt{mtw2}, obteniendo que el valor más adecuado para optimizar la discriminación es 120 GeV. A este caso se le referirá como '2D'\\ 

Por otro lado, si el proceso se invierte y se comienza con un primer corte en \texttt{mtw2}, se encuentra que el umbral óptimo para esta variable es 120 GeV, seguido de un corte en \texttt{mpmet} de 100 GeV. Con esto, se encuentra que un corte en \texttt{mpmet} de 120 GeV optimizaría la señal. A este caso se le referirá como '2D reverso' \\

Estos resultados permiten explorar distintas estrategias para optimizar la selección, dependiendo del orden en el que se realicen los cortes y el impacto que estos tengan en la relación entre señal y fondo.\\ \newpage

\begin{figure}[h!]
     \centering
     \begin{subfigure}[b]{0.43\textwidth}
         \centering
         \includegraphics[width=\textwidth]{images/emu__tttDM_FoM_mpmet_mchi1_mphi100_2D_mpmet110mtw100.png}
         \caption{Caso 2D para la Variable \texttt{mpmet}}
         \label{fig:FoM_2D_mpmet_110_mtw2_100_mpmet}
     \end{subfigure}
     \begin{subfigure}[b]{0.43\textwidth}
         \centering
         \includegraphics[width=\textwidth]{images/emu__tttDM_FoM_mtw2_mchi1_mphi100_2D_mpmet110mtw100.png}
         \caption{Caso 2D para la variable \texttt{mtw2}}
         \label{fig:FoM_2D_mpmet_110_mtw2_100_mtw2}
     \end{subfigure}
     \hfill
          \begin{subfigure}[b]{0.43\textwidth}
         \centering
         \includegraphics[width=\textwidth]{images/emu__tttDM_FoM_mpmet_mchi1_mphi100_2D_mtw2_120_mpmet_100.png}
         \caption{Caso 2D reverso para la variable \texttt{mpmet}}
         \label{fig:FoM_2D_mtw2_120_mpmet_100_mpmet}
     \end{subfigure}
          \begin{subfigure}[b]{0.43\textwidth}
         \centering
         \includegraphics[width=\textwidth]{images/emu__tttDM_FoM_mtw2_mchi1_mphi100_2D_mtw2_120_mpmet_100.png}
         \caption{Caso 2D reverso para la variable \texttt{mtw2}}
         \label{fig:FoM_2D_mtw2_120_mpmet_100_mtw2}
     \end{subfigure}
     \caption{Gráficas representando la figura de mérito una vez se han calculado los cortes de manera secuencial. Arriba se muestra el suceso 2D (cortando primero en \texttt{mpmet} y después en \texttt{mtw2}) y abajo el suceso 2D reverso.}
     \label{fig:FoM_2D}
\end{figure}

Se puede observar que realizar cortes en una variable y después en la otra, y viceversa, varía un poco los resultados del análisis. Se puede observar una ligera diferencia en la forma de la figura de mérito entre las figuras \ref{fig:FoM_2D_mpmet_110_mtw2_100_mpmet} y \ref{fig:FoM_2D_mtw2_120_mpmet_100_mpmet} ya que esta última disminuye de una manera ligeramente más suave que en el caso 2D. Esto es esperable ya que al realizar el corte en la primera variable la distribución de los eventos cambia.\\
