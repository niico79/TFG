
\section{Modelo Estándar de la Física de Partículas}

La física de partículas se fundamenta en lo que se conoce como el Modelo Estándar de la Física de Partículas. Ésta es una teoría que fue desarrollada durante la década de los años 70 por la comunidad científica gracias a los conocimientos previos que se tenían sobre la estructura de la materia, la mecánica cuántica y la relatividad general. El Modelo Estándar está fundamentado en la mecánica cuántica y en la teoría cuántica de campos y explica a la vez la cromodinámica cuántica (relacionada con la interacción nuclear fuerte) y la teoría electrodébil (relacionada con las interacciones electromagnéticas y débil). Sin embargo, falla en explicar la teoría de la relatividad general. \\

% ES IMPORTANTE ESTO? -->
El Modelo Estándar se fundamenta en las simetrías ya que determinan las interacciones y las propiedades de las partículas fundamentales. Esto nace del Teorema de Noether, que indica que a cada simetría continua y derivable de un sistema le corresponde una cantidad conservada. \cite{noether_th} Las simetrías que aparecen en el Modelo Estándar se conocen como simetrías gauge, que indica que la ecuaciones sean invariantes baja un tipo de transformaciones locales. Matemáticamente, el Modelo Estándar se define bajo el grupo de simetría $SU(3) \times SU(2)_L \times U(1)_Y$ que corresponde a la teoría de QCD y a la teoría electrodébil. Del grupo $U(1)_Y$, que es el correspondiente a la hipercarga débil, surge la conservación de carga eléctrica. Del grupo de interacción débil $SU(2)_L$ surge la conservación del isospin débil. Por último, del grupo de la interacción fuerte $SU(3)$ surge la conservación de la carga de color.\\

La teoría del Modelo Estándar defiende que existen un grupo de partículas que constituyen la materia, llamados fermiones, y de las que todo el universo es constituido y que interaccionan entre ellas gracias al intercambio de unas partículas llamadas bosones gauge. Vayamos desglosando poco a poco estas partículas:\\

 Como ya se ha mencionado, los fermiones son las partículas que forman toda la materia: desde las células hasta las estrellas. Los fermiones son 12 partículas con spin semientero (y sus correspondientes antipartículas) que se pueden dividir en dos grupos de 6 partículas: leptones y quarks. Los leptones son partículas fermiónicas sin carga de color. Este grupo se divide en tres subgrupos o generaciones: el electrón $e^-$ y neutrino electrónico $\nu_e$, el muón $\mu$ y el neutrino muónico $\nu_\mu$ y el tau $\tau$ y el neutrino tauónico $\nu_\tau$. El electrón, muón y tau tienen carga eléctrica negativa por lo que interactúan mediante la fuerza electromagnética además de la fuerza débil, mientras que los neutrinos son partículas de carga neutra que interactúan únicamente mediante la fuerza débil. Por otro lado se tienen los quarks, que son partículas con carga de color por lo que interacción mediante la fuerza fuerte. Se dividen en tres generaciones: quarks \textit{up} $u$ y \textit{down} $d$, \textit{charm} $c$ y \textit{strangeness} $s$, \textit{top} $t$ y \textit{bottom} $b$. Los quarks interactúan, aparte de mediante la fuerza fuerte, mediante la interacción electromagnética y la interacción débil. \\

 Por otro lado, los bosones de gauge son partículas con spin entero. Estas partículas son las que portan las fuerzas que sufren todas las partículas. Estos bosones son el fotón $\gamma$ que media la fuerza electromagnética, el gluón $g$ que media la fuerza fuerte y los bosones $W^+$, $W^-$ y $Z^0$ que median la fuerza débil, todos ellos con spin 1. Además, se tiene el bosón de Higgs $H^0$ que se asocia al mecanismo que da masa a las partículas.  \\

\textcolor{red}{EXPLICAR SIMETRÍAS}\\
Además, se explican a continuación las interacciones fundamentales que describe el modelo estándar:\\

De más fuerte a más débil, la \textbf{fuerza nuclear fuerte} surge de la simetría $SU(3)$ y es la responsable de que los nucleones se mantengan ligados en el núcleo. Los mediadores de esta fuerza son los gluones, partícula con masa nula pero que tienen carga de color, lo que les permite interactuar entre ellos. Este tipo de fuerza depende linealmente de la distancia, esto es, el acoplamiento de las partículas con carga de color aumenta con la distancia. Esto puede provocar que si dos quarks se separan mucho, la fuerza sea tan intensa que se produzca un gluón que decae en un par quark-antiquark que se recombinan con los quarks que existían inicialmente. Este es el principio fundamental del confinamiento de color y cuya consecuencia es la aparición de jets hadrónicos que son detectados por los experimentos del CERN. A continuación está la \textbf{fuerza nuclear débil}, siguiendo la simetría $SU(2)$ que se acopla a la parte leptónica del modelo estándar. Está mediada por tres bosones masivos, dos de ellos cargados, el $W^+$ y el $W^-$, y un bosón sin carga, el $Z^0$. El hecho de que los bosones tengan masa hace que sea de corta distancia y que las partículas que interactúan mediante esta fuerza decaigan muy rápidamente. Además, la interacción débil y la electromagnética están acopladas en los que se conoce como Teoría Electrodébil. La última interacción que describe el modelo estándar es la \textbf{interacción electromagnética}, regida por el grupo de simetría $U(1)$, actúa entre partículas con carga eléctrica. Su partícula mediadora, el fotón $\gamma$, no tiene masa por lo que el rango de esta interacción es infinito. Esta fuerza es la responsable de los fenómenos eléctricos y magnéticos que vivimos en el día a día. \\

Aunque no sea descrito por el Modelo Estándar, es de interés explicar la fuerza gravitatoria. Esta fuerza no está mediada por ninguna partícula sino que es la consecuencia de vivir en un Espacio-Tiempo que se curva debido a la presencia de materia. La gravedad tiene un papel fundamental en la búsqueda de la Materia Oscura ya que se sabe que interacciona mediante esta fuerza.\\

Por otro lado, y aunque no sea una fuerza fundamental, cabe destacar el mecanismo de Higgs que es el que provoca que las partículas adquieran masa. Esta teoría fue propuesta por Peter Higgs, Robert Brout y François Englert y predice la existencia de un campo que surge de la ruptura del grupo de simetría $SU(2)_LU(1)_Y$, correspondiente a la teoría electrodébil. Este campo, llamado campo de Higgs, interactúa con algunas partículas infiriéndoles masa. \textcolor{red}{REVISAR}

\section{Materia Oscura}

La búsqueda de Materia Oscura (DM por sus siglas en inglés, \textit{Dark Matter}) es uno de los proyectos más ambiciosos en el campo de la física actualmente. La materia oscura se predice que es casi 5 veces más abundante que la materia bariónica, que es el tipo de materia a la que estamos acostumbrados y que compone todo desde las estrellas hasta los núcleos de los átomos. El problema de detectar la materia oscura no nace de su abundancia, sino de que no interacciona mediante la fuerza electromagnética provocando que ni emita ni refleje luz. Por tanto, la tarea de detección de este tipo de materia es muy compleja y la mayoría de las veces se tiene que inferir de efectos gravitaciones sobre la materia bariónica.\\

Aunque este tipo de materia no pueda ser detectada de manera directa, se puede inferir gracias a ciertos fenómenos gravitacionales que se observan. Este concepto lo introdujo el astrónomo suizo Fritz Zwicky, que estimó la masa del cúmulo de Coma mediante el teorema del virial. La diferencia entre la masa estimada y la observada era de 400 veces, lo que le llevó a Zwicky a postular que existía un tipo de materia que no observaríamos que daría al cúmulo de masa \cite{Zwicky}. 40 años después, Vera Rubin midió la curva de rotación de las galaxias espirales. Su decubrimiento nace de que vió que todas las estrellas giran a una velocidad casi constante, lo que indica que más allá del bulbo galáctico tiene que existir una materia que aporte masa \cite{rubin}. A partir de entonces, la comunidad científica empezó a centrarse en descubrir la naturaleza de este tipo de materia.\\

Se ha propuesto varios candidatos para materia oscura, pero en este análisis se va a hacer hincapié en los \textit{WIMPs}. Los \textit{WIMPs} (\textit{Weakly Interacting Massive Particles} por sus siglas en inglés) son unas hipotéticas partículas que interactuarían mediante la fuerza gravitatoria y mediante otras fuerzas no descubiertas todavía a escala de la fuerza nuclear débil. Se puede teorizar algunas propiedades de estas nuevas partículas: deben tener una carga neutra (ya que si tuviesen carga interactuarían mediante la fuerza electromagnética) y tienen que ser masivas porque no han sido vistas en experimentos a energías bajas y medias.  Si estas partículas se produjesen en el \textit{LHC} sería posible detectarlas de manera indirecta, de la misma manera que se hace con los neutrinos. En este análisis, el objetivo principal es identificar indicios de una posible interacción entre una partícula del Modelo Estándar y una partícula de Materia Oscura. Este choque daría lugar a una característica distintiva en los eventos registrados: una cantidad significativa de energía faltante, atribuida a la presencia de una partícula de Materia Oscura $\chi$ que escapa sin ser detectada.\\

PSEUDOESCALAR Y ESCALAR
 \textcolor{red}{PONER MUCHO MÁS} 

\section{Procesos relacionados con quarks top}

En distintos estudios se ven diferentes procesos de producción de las partículas de Materia Oscura, en este análisis se verá la producción de Materia Oscura con pares quarks top-antitop $t\overline{t}$ + DM o con un único quark top o antitop $t/\overline{t} $ + DM. En este apartado se explicará brevemente las diferentes interacciones y procesos en los que interviene el quark top.\\

El quark top es la partícula más masiva del Modelo Estándar, por tanto hace falta mucha energía para producirla. Es debido a esto que además decae en un tiempo ínfimo y se infiere detectando los productos de su integración.

\subsection{Interacción $t\bar{t}$ descrita por el Modelo Estándar}

Un par top-antitop se puede crear unicamente mediante la fuerza fuerte, decayendo a partir de un gluón muy energético. El quark top decae preferentemente mediante la interacción electrodébil en un bosón W y un quark \textit{bottom}, \textit{down} o \textit{strange}. Sin embargo, el modo de decaimiento más probable es en un par bosón $W$ con quark $b$ debido a que los elementos relacionados con los quarks $d$ y $s$ en la matriz CKM son pequeños. \footnote{La matriz CKM (Cabibbo-Kobayashi-Maskawa) contiene la información sobre cómo cambian de sabor los quarks al sufrir interacción débil.} Por tanto, se estudiará los procesos con desintegración $t \rightarrow W^+b $. El bosón $W^+$ no es una partícula estable, por lo que decae en un par leptón-neutrino $W^+ \rightarrow \ell^+ \nu_\ell$ (siendo $\ell$ un electrón, muón o tau). Por tanto, la señal que se observa es $t \rightarrow \ell^+\nu_\ell b$. De manera análoga el quark antitop decae como $\bar{t} \rightarrow \ell^- \bar{\nu_\ell}\bar{b}$. Por tanto, los sucesos estándar de las interacciones top-antitop sufre la siguiente desintegración: $t\bar{t} \rightarrow \ell^+  \ell^- \nu_{\ell} \bar{\nu_\ell} b \bar{b}$. \cite{topdecays} A continuación se muestran los diagramas de Feynmann relacionados con la desintegración de este par de quarks
\url{https://www.phy.bnl.gov/~partsem/fy09/TTait_Talk_06_18_09.pdf}

\begin{tikzpicture}
\begin{feynman}
\diagram [horizontal=a to b] {
  i1 [particle=\(g\)] -- [gluon] a -- [gluon] i2 [particle=\(g\)],
  a -- [fermion] b,
  f1 [particle=\(t\)] -- [fermion] b -- [fermion] f2 [particle=\(\bar{t}\)],
};
\end{feynman}
\end{tikzpicture}


\begin{figure}
    \centering
    \includegraphics[width=0.5\linewidth]{images/ctg_graph_big.png}
    \caption{Caption}
    \label{fig:enter-label}
\end{figure}